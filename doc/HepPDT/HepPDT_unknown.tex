\section { Handling Unknown Particle ID's }
\label{unknownID}

\subsection { Abstract Base Class }

The arguments for processUnknownID() must be set so that any
inheriting class has all necessary information. 
We pass a const reference to the ParticleDataTable instance.  

\begin{verbatim}
namespace HepPDT {

// forward declaration to avoid circular dependencies
class ParticleDataTable;

class ProcessUnknownID  {
 
public:

  /// safety wrapper to avoid secondary calls to processUnknownID
  ParticleData  * callProcessUnknownID( ParticleID, const ParticleDataTable & );

  /// allow cleanup by ParticleDataTable
  virtual ~ProcessUnknownID( ) {}


protected:
  ProcessUnknownID( ) : alreadyHere(false) {}

private: 

  bool    alreadyHere;
  virtual ParticleData  * processUnknownID( ParticleID, 
                                            const ParticleDataTable & ) = 0;

};  // ProcessUnknownID

}       // HepPDT
\end{verbatim}


\subsection { SimpleProcessUnknownID }

This is the default implementation of processUnknownID().
Notice that it simply returns a null pointer.

\begin{verbatim}
namespace HepPDT {

class SimpleProcessUnknownID : public ProcessUnknownID {
public:
  SimpleProcessUnknownID() {}

  virtual ParticleData  * processUnknownID
                      ( ParticleID key, const ParticleDataTable & pdt )
    { return 0; }
    
};

}       // HepPDT
\end{verbatim}


\subsection { HeavyIonUnknownID }

HeavyIonUnknownID creates and returns a pointer to a ParticleData.
This processUnknownID method only uses the proton mass to calculate
an approximate mass for the nuclear fragment.

\begin{verbatim}
// HeavyIonUnknownID.hh
namespace HepPDT {

class HeavyIonUnknownID : public ProcessUnknownID {
public:
  HeavyIonUnknownID() {}

  virtual ParticleData  * processUnknownID( ParticleID,  const ParticleDataTable & pdt );
    
};

}       // HepPDT

\end{verbatim}

\begin{verbatim}
// HeavyIonUnknownID.cc
namespace HepPDT {

ParticleData * HeavyIonUnknownID::processUnknownID
              ( ParticleID key, const ParticleDataTable & pdt ) 
{ 
    if( ! key.isNucleus() ) return 0;
     
    // have to create a TempParticleData with all properties first
    TempParticleData tpd(key);
    // calculate approximate mass
    // WARNING: any calls to particle() from here MUST reference 
    //          a ParticleData which is already in the table
    // This convention is enforced.
    const ParticleData * proton = pdt.particle(2212);
    if( ! proton ) return 0;
    double protonMass = proton->mass();
    tpd.tempMass = Measurement(key.A()*protonMass, 0.);
    return new ParticleData(tpd);
}

}       // HepPDT

\end{verbatim}


\subsection { Using MyProcessUnknownID }

To use HeavyIonUnknownID, you simply need to specify it when you
create your ParticleDataTable object: 
\begin{verbatim}
#include "HepPDT/HeavyIonUnknownID.hh"
...
HepPDT::ParticleDataTable pdt( "Handle Heavy Ions", 
                                new HepPDT::HeavyIonUnknownID );
\end{verbatim}

You may also create your own method and call it in the same way.


\vfill\eject
