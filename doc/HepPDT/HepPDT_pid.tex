\section {ParticleID.hh}
\label{PID}

\begin{tabbing}

{\bf namespace HepPDT} \\  \\

{\bf Free functions:} \\
\hspace{0.5in}  {\bf double spinitod( int js ); } \\
\hspace{0.5in}  {\bf int spindtoi( double spin ); } \\  \\

{\bf Public members:} \\
\hspace{0.5in}  {\bf enum location } 
       $\{$ nj=1, nq3, nq2, nq1, nl, nr, n, n8, n9, n10 $\}$; \\
\hspace{0.5in}  {\bf  struct Quarks } $\{$ \\
\hspace{1.0in}    Quarks( ) : nq1(0), nq2(0), nq3(0)  $\{\}$ \\
\hspace{1.0in}    Quarks( short q1, short q2, short q3) : nq1(q1), nq2(q2), nq3(q3) $\{\}$ \\
\hspace{1.0in}    short nq1;
    short nq2;
    short nq3; $\}$; \\  \\

{\bf CLASS ParticleID } \\  \\

{\bf Public Methods:} \\
\hspace{0.5in}  {\bf ParticleID( int pid = 0 ); } \\
\hspace{1.0in}  The constructor.\\
\hspace{0.5in}  {\bf ParticleID( const ParticleID \& orig ); } \\
\hspace{1.0in}  The copy constructor. \\
\hspace{0.5in}  {\bf ParticleID \& operator=( const ParticleID \& ); } \\
\hspace{1.0in}  The assignment constructor. \\
\hspace{0.5in}  {\bf void swap( ParticleID \& other ); } \\
\hspace{1.0in}  The swap constructor. \\
\hspace{0.5in}  {\bf bool  operator $<$  ( ParticleID const \& other ) const;} \\
\hspace{1.0in}  Comparison operator. \\
\hspace{0.5in}  {\bf bool  operator == ( ParticleID const \& other ) const;} \\
\hspace{1.0in}  Equality operator. \\
\hspace{0.5in}  {\bf int    pid( )        const; } \\
\hspace{1.0in}  Returns the PID. \\
\hspace{0.5in}  {\bf int abspid( )        const; } \\
\hspace{1.0in}  Returns the absolute value of the PID. \\
\hspace{0.5in}  {\bf bool isValid( )   const; }\\
\hspace{1.0in}  Returns true if this integer obeys the numbering scheme rules. \\
\hspace{0.5in}  {\bf bool isMeson( )   const; }\\
\hspace{1.0in}  Returns true if this integer obeys the meson  portion of the numbering scheme rules\\
\hspace{0.5in}  {\bf bool isBaryon( )  const; }\\
\hspace{1.0in}  Returns true if this integer obeys the baryon portion of the numbering scheme rules.\\
\hspace{0.5in}  {\bf bool isDiQuark( ) const; }\\
\hspace{1.0in}  Returns true if this integer obeys the diquark portion of the numbering scheme rules.\\
\hspace{0.5in}  {\bf bool isHadron( )  const; }\\
\hspace{1.0in}  Returns true if either isBaryon or isMeson is true. \\
\hspace{0.5in}  {\bf bool isLepton( )  const; }\\
\hspace{1.0in}  Returns true if the fundamentalID is 11-18. \\
\hspace{0.5in}  {\bf bool isNucleus( )  const; }\\
\hspace{1.0in}  Returns true if this integer obeys the ion numbering scheme rules. \\
\hspace{0.5in}  {\bf bool isPentaquark( )  const; }\\
\hspace{1.0in}  Returns true if this integer obeys the pentaquark numbering scheme rules. \\
\hspace{0.5in}  {\bf bool isSUSY( )  const; }\\
\hspace{1.0in}  Returns true if this integer obeys the SUSY numbering scheme rules. \\
\hspace{0.5in}  {\bf bool isRhadron( )  const; }\\
\hspace{1.0in}  Returns true if this integer obeys the R-hadron numbering scheme rules. \\
\hspace{0.5in}  {\bf bool isDyon( )  const; }\\
\hspace{1.0in}  Returns true if this integer obeys the dyon numbering scheme rules. \\
\hspace{0.5in}  {\bf bool isQBall( )  const; }\\
\hspace{1.0in}  Returns true if this integer obeys the ad-hoc Q-ball numbering scheme rules. \\
\hspace{0.5in}  {\bf bool hasUp( )      const; }\\
\hspace{1.0in}  Returns true if this is a valid PID and it has an up quark. \\
\hspace{0.5in}  {\bf bool hasDown( )    const; }\\
\hspace{1.0in}  Returns true if this is a valid PID and it has a down quark.\\
\hspace{0.5in}  {\bf bool hasStrange( ) const; }\\
\hspace{1.0in}  Returns true if this is a valid PID and it has a strange quark.\\
\hspace{0.5in}  {\bf bool hasCharm( )   const; }\\
\hspace{1.0in}  Returns true if this is a valid PID and it has a charm quark.\\
\hspace{0.5in}  {\bf bool hasBottom( )  const; }\\
\hspace{1.0in}  Returns true if this is a valid PID and it has a bottom quark.\\
\hspace{0.5in}  {\bf bool hasTop( )     const; }\\
\hspace{1.0in}  Returns true if this is a valid PID and it has a top quark.\\
\hspace{0.5in}  {\bf int  jSpin( )        const; }\\
\hspace{1.0in}  jSpin returns 2J+1, where J is the total spin \\
\hspace{0.5in}  {\bf int  sSpin( )        const; }\\
\hspace{1.0in}  sSpin returns 2S+1, where S is the spin \\
\hspace{0.5in}  {\bf int  lSpin( )        const; }\\
\hspace{1.0in}  lSpin returns 2L+1, where L is the orbital angular momentum \\
\hspace{0.5in}  {\bf int fundamentalID( ) const; }\\
\hspace{1.0in}  Returns the first 2 digits if this is a valid PID and it is neither
                neither a meson, a baryon, \\
\hspace{1.0in}	nor a diquark.  If this is a meson, baryon, or
		diquark, fundamentalID returns zero. \\
\hspace{0.5in}  {\bf int extraBits( ) const; }\\
\hspace{1.0in}  Returns any digits beyond the 7th digit 
                (e.g. outside the numbering scheme). \\
\hspace{0.5in}  {\bf Quarks quarks( ) const; }\\
\hspace{1.0in}  Returns a struct with the 3 quarks. \\
\hspace{0.5in}  {\bf int threeCharge( ) const; }\\
\hspace{1.0in}  Returns 3 times the charge, as inferred from the quark content.\\
\hspace{0.5in}  {\bf double charge( ) const; }\\
\hspace{1.0in}  Returns the actual charge, as inferred from the quark content.\\
\hspace{1.0in}  If the fundamentalID is non-zero, then a lookup table is used. \\
\hspace{0.5in}  {\bf int A( ) const; }\\
\hspace{1.0in}  If this is an ion, returns A.\\
\hspace{0.5in}  {\bf int Z( ) const; }\\
\hspace{1.0in}  If this is an ion, returns Z.\\
\hspace{0.5in}  {\bf int lambda( ) const; }\\
\hspace{1.0in}  If this is an ion, returns nLambda.\\
\hspace{0.5in}  {\bf unsigned short digit(location) const; }\\
\hspace{1.0in}  digit returns the base 10 digit at a named location in the PID \\
\hspace{0.5in}  {\bf const std::string PDTname() const; }\\
\hspace{1.0in}  Returns the HepPDT standard name. \\ \\

{\bf Private Members:} \\
\hspace{0.5in}  {\bf int itsPID; } \\

\end{tabbing}

\vfill\eject

